\documentclass[../../../main]{subfiles}
\begin{document}

\section{方法}

図\ref{fig:electric-servo-mechanism}のような実験装置を組み立てた。
この実験装置をブロック線図で書いたものが、
図\ref{fig:servo-block-line-detail}である。

\subsection{実験1}
実験1では、SW1、SW2のスイッチをオンにした。
適当な$\theta_1$に対して、$k_p$のメモリを$60$, $80$, $100$と変化させ、
その$k_p$に対してそれぞれ$k$のメモリも$0$, $60$, $80$, $100$と変化させ、
計12個の組み合わせで偏差電圧$e$をオシロスコープを用いて測定した。

\subsection{実験2}
実験2から、SW2はオフにする。
ポテンショメーター2に繋がる歯車を外し、Pゲインが$0$となるようにした。
\footnote{
	フィードバックからフィードフォーワードに変化させた。
}
次にオシロスコープの端子をDゲインの入力、すなわち$e_t$に接続しその波形を観察した。
撮影した画像をもとに、時定数$T$を求めた。

\subsection{実験3}
実験2と同様SW2はオフにし、Pゲインは$0$とした。
まず、2台の電圧計を前置増幅器の前後に接続し$k_1$を$60$, $80$, $100$と変化させ、
それに対応する$e$, $e_p$を計測した。
その比$e_p/e$から$k_p$の値を求めた。
同時に、回転体の回転速度を回転速度系にて計測し、$\dot{\theta_2}$を求めた。
次に1台の電圧計は前置僧服機の前、もう1台はDゲインの出力に接続し、
$k$のメモリを$60$, $80$, $100$と変化させて、$\dot{\theta_2}$と電圧$e_t$の比から$k$を求めた。
最後に、$\dot{\theta_2}$の比と$e_p$の比からサーボモーターのゲイン$K=k_ak_m$を求めた。


\end{document}
